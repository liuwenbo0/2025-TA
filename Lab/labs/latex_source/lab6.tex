\documentclass[12pt,a4paper]{article}
%-- coding: UTF-8 --
\usepackage[UTF8]{ctex}
\usepackage[utf8]{inputenc}
\usepackage{geometry}
\usepackage{graphicx} % 引入图片
\usepackage{enumitem} % 取消列表默认间距
\geometry{left=3.18cm,right=3.18cm,top=2.54cm,bottom=2.54cm}
\usepackage{hyperref}
\hypersetup{hidelinks,
	colorlinks=true,
	allcolors=black,
	pdfstartview=Fit,
	breaklinks=true}
\usepackage{listings}
\usepackage{xcolor}
\usepackage{fontspec}
\usepackage{booktabs} % 三线表
\usepackage{float}

\usepackage{tikz}
\usepackage{amsmath}
\usepackage{colortbl}
\newcommand\y{\cellcolor{clight2}}
\definecolor{clight2}{RGB}{212, 237, 244}%
\newcommand\tikznode[3][]%
   {\tikz[remember picture,baseline=(#2.base)]
      \node[minimum size=0pt,inner sep=0pt,#1](#2){#3};%
   }
\tikzset{>=stealth}
\renewcommand\vec[1]{\mathbf{#1}}

% 嵌入代码风格
\lstset{
	language    = c++,
	breaklines  = true,
	captionpos  = b,
	tabsize     = 4,
	columns     = fullflexible,
	commentstyle = \color[RGB]{0,128,0},
	keywordstyle = \color[RGB]{0,0,255},
	basicstyle   = \small\ttfamily,
	stringstyle  = \color[RGB]{148,0,209}\ttfamily,
	rulesepcolor = \color{red!20!green!20!blue!20},
	showstringspaces = false,
}


%伪代码
\usepackage{multirow}
\usepackage{algorithm}
\usepackage[noend]{algpseudocode}
\usepackage{amsmath}

\title{实验六 \hspace{0.5cm} 图的最短路径与最小生成树算法}
\author{
\begin{tabular}{c @{\hspace{5mm}} c}
    黄韦杰 & 刘嘉杰 \\  % 作者名
    \href{mailto:hwj@hust.edu.cn}{hwj@hust.edu.cn} & \href{mailto:m202474039@hust.edu.cn}{m202474039@hust.edu.cn} % 邮箱
\end{tabular}
}
\date{November, 2024}

\begin{document}
\maketitle

\section{前言}
1736年29岁的欧拉向圣彼得堡科学院递交了《哥尼斯堡的七座桥》的论文,在解答问题的同时,开创了数学的一个新的分支——图论与几何拓扑,也由此展开了数学史上的新历程。七桥问题提出后,很多人对此很感兴趣,纷纷进行试验,但在相当长的时间里,始终未能解决。欧拉通过对七桥问题的研究,不仅圆满地回答了哥尼斯堡居民提出的问题,而且得到并证明了更为广泛的有关一笔画的三条结论,人们通常称之为“欧拉定理”。

图模拟一组连接,由节点和边组成。一个图就是一些顶点的集合,这些顶点通过一系列边结对(连接)。顶点用圆圈表示,边就是这些圆圈之间的连线,顶点之间通过边连接。图计算是研究客观世界当中的任何事物和事物之间的关系,对其进行完整的刻划、计算和分析的一门技术。图计算技术主要是由点和边来组成的。举例来说,点可以是坐在演播室里的三个人,这三个人就是三个点,所谓的边就是这三个人之间的关系。比如说,同事关系、亲戚关系、夫妻关系等,而且两个人之间的关系不仅限于一种描述,还可能有一些其他的关系,比如说项目合作关系、投资关系等。原则上两点之间的关系没有上限,可以有很多的各种关系。点越多,当然各点相互之间可能的关系也就越错综复杂。

图 $G$ 由节点 $V$(vertice)与边 $E$(edge)构成,我们一般表示为 $G(V,E)$。图数据的典型例子比如网页链接关系、社交网络、商品推荐等。比如微信的社交网络,是由节点(个人、公众号)和边(关注、点赞)构成的图;淘宝的交易网络,是由节点(个人、商品)和边(购买、收藏)构成的图。如此一来,抽象出来的图数据构成了研究和商用的基础,可以以此探究出“世界上任意两个人之间的人脉距离”,“关键意见领袖”等。将这些应用到商业领域,其底层的运算往往是图相关的算法,这便是图计算。

在金融行业中,图计算以及认知技术重点应用的业务领域包括:金融风险的管控、客户的营销拓展,内部的审计监管、以及投资理财等方面。再如国内商业银行都面临的信贷风险问题:受国内经济下行的影响,企业客户贷款的不良率攀升。为了提高银行对企业不良风险传导的预测能力,利用图计算和图认知技术,完整刻画企业客户之间、企业与自然人之间的社会关系、经济往来关系,构建全方位的风险关联网络,实现风险要素的动态性和完整性呈现。图计算模型在大数据公司,尤其是 IT 公司是非常流行的一大模型,它是很多实际问题最直接的解决方法。近几年,随着数据的多样化,数据量的大幅度提升和算力的突破性进展,超大规模图计算在大数据公司发挥着越来越重要的作用,尤其是以深度学习和图计算结合的大规模图表征为代表的系列算法。图计算的发展和应用有井喷之势,各大公司也相应推出图计算平台,例如 Google Pregel、Facebook Giraph、腾讯星图、华为图引擎服务 GES 等。



\section{实验项目结构}

\begin{itemize}[noitemsep]
    \item[$-$] 1\_shortest\_path \textit{题目一 \hspace{0.1cm} 最短路径问题}
    \begin{itemize}[noitemsep]
        \item[$-$] include
        \begin{itemize}[noitemsep]
            \item[$\bullet$] util.hpp \textit{常用函数头文件}
            \item[$\bullet$] Solution.hpp \textit{待完成}
        \end{itemize}
        \item[$-$] data
        \item[$\bullet$] main.cpp \textit{主程序代码}
    \end{itemize}
    \item[$-$] 2\_minimun\_spanning\_tree \textit{题目二 \hspace{0.1cm} 最小生成树问题}
    \begin{itemize}[noitemsep]
        \item[$-$] 略
    \end{itemize}
    \item[$-$] 3\_minimun\_spanning\_tree\_plus \textit{题目三 \hspace{0.1cm} 最小生成树问题 Plus(拓展题)}
    \begin{itemize}[noitemsep]
        \item[$-$] 略
    \end{itemize}
    \item[$-$] 4\_building\_roads \textit{题目四 \hspace{0.1cm} 道路建设(拓展题)}
    \begin{itemize}[noitemsep]
        \item[$-$] 略
    \end{itemize}
\end{itemize}

\textcolor{red}{请注意,每次修改完代码之后,需要重新编译运行 main.cpp,如果直接执行上次编译好的 main.exe 或 main,新的修改将不会生效。在本地测试通过后请将代码提交到 OJ 上,实验内容以OJ平台为主。}

\section{实验内容}

\subsection{最短路径问题}
给定 $n$ 个点( $1\sim n$ 表示),$m$ 条边构成的无向连通图(任意两点互相可达)。第 $i$ 条边 $edges[i] = \{x, y, z\}$ 表示 $x$ 与 $y$ 之间有一条长度为 $z$ 的无向边。请你求出从 $1$ 号点到 $n$ 号点的最短路距离。


\begin{lstlisting}
    class Solution {
    public:
        long long shortest_path(int n, vector<vector<int>>& edges) {
            // 请在这里完成你的代码
        }
    };
\end{lstlisting}

\textbf{注:请使用 Dijkstra算法 求解。}

数据范围:100\% 的数据满足:$1\le n \le 10^3, 1 \le m \le 10^5, 1 \le z \le 10^9$



\subsection{最小生成树问题}
给定 $n$ 个点( $1\sim n$ 表示),$m$ 条边构成的无向连通图(任意两点互相可达)。第 $i$ 条边 $edges[i] = \{x, y, z\}$ 表示 $x$ 与 $y$ 之间有一条长度为 $z$ 的无向边。请你求出最小生成树的边权和。

\begin{lstlisting}
    class Solution {
    public:
        long long minimum_spanning_tree(int n, vector<vector<int>>& edges) {
            // 请在这里完成你的代码
        }
    };
\end{lstlisting}

\textbf{注:请使用 Prim算法 求解。}

数据范围:100\% 的数据满足:$1\le n \le 10^3, 1 \le m \le 10^5, 1 \le z \le 10^9$




\section{实验思考}

\begin{enumerate}
    \item 你所使用的 Dijkstra 算法的时间复杂度是多少?还有可能的优化空间吗?
    \item 如果要输出具体的最短路径(包括路径上的点和路径上的边),你的代码应该如何修改?(仅说明需要哪些新的数组以及如何使用即可,不需要重写代码)
    \item 你所使用的 Prim 算法的时间复杂度是多少?还有可能的优化空间吗?
    \item 如果要输出具体的最小生成树,你的代码应该如何修改?(仅说明需要哪些新的数组以及如何使用即可,不需要重写代码)
    
\end{enumerate}

\section{拓展实验}

\subsection{题目 I 最短路径 Plus}

给定 $n$ 个点($1 ∼ n$ 表示),$m$ 条边构成的图。

第 $i$ 条边 $edges[i] = \{x, y, z\}$ 表示 $x$ 与 $y$ 之间有一条长度为 $z$ 的无向边。请你求出从 $1$ 号点到各点的最短路距离。

数据范围:100\%的数据满足:$1\leq n \leq 10^5, 1\leq m \leq 10^6, 1\leq z\leq 10^9$

格式:

Input

第一行为两个正整数$n,m$,分别表示点的数目和边的数目

接下来$m$行,每行三个非负整数$u_i, v_i, w_i$,表示 $u_i$ 到 $v_i$ 有一条长度为 $w_i$ 的无向边。

接下来一个整数 $r$,表示询问的次数

最后一行为 $r \space (1 \leq r \leq 10^5)$ 个整数 $q_i \space (1 \leq q_i \leq n)$,表示询问从 $1$ 号点到 $q_i$ 号点的最短路距离,如果从 $1$ 号点不能到达 $q_i$ 号点,输出 $-1$

Output

输出一行,共 $r$ 个整数,表示对于每个询问的回答

\subsection{题目 II 最小生成树 Plus}

给定 $n$ 个点( $1\sim n$ 表示),$m$ 条边构成的无向连通图(任意两点互相可达)。第 $i$ 条边 $edges[i] = \{x, y, z\}$ 表示 $x$ 与 $y$ 之间有一条长度为 $z$ 的无向边。请你求出最小生成树的边权和。

\begin{lstlisting}
    class Solution {
    public:
        long long minimum_spanning_tree(int n, vector<vector<int>> edges) {
            // 请在这里完成你的代码
        }
    };
\end{lstlisting}

数据范围:100\% 的数据满足:$1\le n \le 10^5, 1 \le m \le min(\frac{n(n-1)}{2}, 10^6), 1 \le z \le 10^9$

\textbf{注:请使用 Kruskal算法 求解。}如果你不清楚如何对边按照边权进行排序,请参考 lab5 实验指导书的附录部分

\subsection{题目 III 吃豆人}

吃豆人的使命是吃掉迷宫内所有的豆子,而且不能被鬼魂抓到。然而,当迷宫里没有豆子时,吃豆人们必须立刻按下机关离开迷宫。

迷宫用 $a \times b$ 的网格表示,其中每个地图格代表的内容如下所示:
- `-` 代表空的格子
- `@` 代表障碍物
- `P` 代表吃豆人
- `E` 代表鬼魂

吃豆人或者鬼魂可以从他们所在的格子走到相邻的没有障碍物的格子中。作为玩家的你,可以在空格子上放置障碍物,帮助吃豆人们。

每个吃豆人必须走到 $(a, b)$ 处按下机关,同时你需要保证任何鬼魂都不能接触到机关。

格式:

Input
第一行为一个整数 $q \ (1 \leq q \leq 100)$,代表询问的个数。接下来有 $q$ 组询问:

每组询问中,第一行包含两个整数 $a, b \ (1 \leq a, b \leq 200)$

接下来 $a$ 行,每行有 $b$ 个字符,如题目描述所示。

Output
一共有 $q$ 行,对于每一个询问,如果可以满足题目中的要求,输出 `Pac-Man Win`,否则输出 `Pac-Man Lose`

\subsection{题目 IV 道路建设}

市长 Bob 所管理的城市由 $n$ 个社区(编号 $0\sim n-1$)组成,由于疫情封控需要保证每个社区的物资供应。在第 $i$ 个社区建立物资供应站需要花费 $w_i$ 元。

为了节省建立物资供应站的经费开销,Bob 提供了 $m$ 个道路建设方案,第 $i$ 个方案 $\{x, y, z\}$ 表示在 $x$ 社区向 $y$ 社区修一条路需要花费 $z$ 元。只要每个社区能够收到物资供应,就符合要求。Bob 保证这 $m$ 条道路能够使 $n$ 个社区连通,请你帮忙求出最经济的建设方案。

提示:只要社区建立了物资供应站,或者与某个建立了物资供应站的社区相连通,那么该社区就可以收到物资供应。
\begin{lstlisting}
    class Solution {
    public:
        long long build_roads(int n, vector<int>& w, vector<vector<int>>& edges){
            // 请在这里完成你的代码
        }
    };
\end{lstlisting}
\begin{figure}[h]
    \centering
    \includegraphics[width=12cm]{img/lab6/build_roads.pdf}
    \caption{道路建设例子}
    \label{fig:build_roads}
\end{figure}

如图 \ref{fig:build_roads} 所示,左侧图表示有 0 - 3 共 4 个社区,建设物资供应站的费用依次为 5,4,4,3 元,另外还有 6 条边使得这 4 个社区连通。右侧图是最省钱的方案,我们选择 3 号社区建立物资供应站,然后选择 3 条边权为 2 的边。


数据范围:100\% 的数据满足:$1\le n \le 10^5, 1 \le m \le min(\frac{n(n-1)}{2}, 10^6), 1 \le w[i], z \le 10^9$

\section{附录}

\subsection{图的存储}

图的表示方法有很多种,按照底层数据结构可以分为顺序表存储和链式表存储。图分为点集和边集两部分,我们一般仅考虑边的存储方式。

设 $n$ 为图的点数,$m$ 为图的边数。

\subsubsection{边集存储}

边集存储是最简单的存储方式,它是一个数组,将所有边依次存放起来。
\begin{lstlisting}
    const int M = 1000000;
    struct Edge{
        int x, y;
    }e[M];
\end{lstlisting}

更为一般的,我们也可以用一个长度为 2 的 vector 数组来表示一条边(有向或无向),然后用一个 vector 数组将这些边存储起来。
\begin{lstlisting}
    vector<vector<int>> edegs = {{1, 2}, {2, 3}, {1, 3}};
    for(auto &edge : edges) {
        cout << edge[0] << ' ' << edge[1] << endl;
    }
\end{lstlisting}

二维数组 edges 包含了 3 条边,每条边 edge 用一个一维数组存放,edge[0] 和 edge[1] 分别是边的两个端点。如果要表示带权图,那么 edge 将是一个长度为 3 的数组,edge[2] 表示对应边的权值。

不论是自定义结构体还是直接使用 vector,他们的空间复杂度都是 $O(m)$,但是想要找到某个点连接的所有边是不容易的,复杂度为 $O(m)$。这种存图方式不经常使用,但不可否认它是所有存图方式中,最简单也最直接的一种方式。

\subsubsection{邻接矩阵}

领接矩阵可以用一个大小为 $n\times n$ 的二维数组表示。在不带权的图中,矩阵元素仅由 0 和 1 构成即可表示整张图。而在带权图中,我们可以提前预设一个值(例如无穷大 inf)表示边不存在,而其他值表示对应边的权值。

\begin{lstlisting}
    const int N = 1000;
    int adj[N][N];
    
    memset(adj, 0x3f, sizeof adj); 
    int n, m;
    cin >> n >> m;
    for(int i = 1; i <= m; i++) {
        int u, v, w;
        cin >> u >> v >> w;
        adj[u][v] = adj[v][u] = w; // 无向带权图
    }
\end{lstlisting}

上述代码展示了如何使用领接矩阵存无向带权图。其中的 memset 函数是指将 adj 数组的每个字节都赋值为 0x3f,因为 int 是 4 个字节,所以最终 adj 中的每个元素都等于 0x3f3f3f3f,这是一个十六进制数字,在十进制下为 1061109567,可以用来表示 inf。

邻接矩阵的优势在于可以 $O(1)$ 的访问一条边,劣势在于空间复杂度为 $O(n^2)$,通常只能保存点数为 $10^3$ 规模的图。(因为$10^6$ 个 int 内存占用接近 4MB)

\subsubsection{邻接表}

通常,邻接表是指 adj[u] 存放了 u 指向的所有点,用链表实现。但其实更简单的方法是一个使用 vector。

\begin{lstlisting}
    int n, m;
    cin >> n >> m;
    vector<vector<pair<int, int>>> adj(n); //创建 n 个二维数组,下标范围 0 ~ n-1
    for(int i = 1; i <= m; i++) {
        int u, v, w;
        cin >> u >> v >> w;
        adj[u].push_back(make_pair(v, w));  // 方式 1
        adj[v].push_back({u, w});           // 方式 2
    }
\end{lstlisting}

上述展示了如何使用 vector 二维数组来表示无向带权图。我们使用 push\_back() 动态的向 vector 中添加元素,这保证了总体空间是 $O(m)$ 的。另外,为了存放边权,我们使用 pair<int,int> 二元组,你可以通过下面的方式访问 u 连接的所有边:

\begin{lstlisting}
    for(int i = 0; i < adj[u].size(); i++) {
        cout << u << ' ' << adj[u][i].first << ' ' << adj[u][i].second << endl;
    }
\end{lstlisting}

如你所见,你可以通过 first 和 second 来分别访问 pair<int,int> 中的第一个和第二个元素。

\subsection{最短路径问题}

最短路径问题有很多,我们这里仅讨论单源最短路径问题,并且仅局限于 Dijkstra 算法来解决该问题。

单源最短路径:给定一个有向图 $G=(V, E)$,节点以 [1,n] 之间的连续整数编号,$(u,v,w)$ 描述一条从 $u$ 出发,到达 $v$,边权为 $w$ 的有向边。设 $1$ 号点为起点,求长度为 $n$ 的数组 $dist$,其中 $dist[i]$ 表示从起点 $1$ 到节点 $i$ 的最短路径长度。

$Dijkstra$ 算法:
\begin{enumerate}[itemsep=0 pt]
    \item 初始化 $dist[1] = 0$,其余节点 $dist$ 值为正无穷大。
    \item 找出一个未标记的、$dist[x]$ 最小的节点 $x$,然后标记 $x$。(对应于 EXTRACT-MIN 操作)
    \item 扫描 $x$ 的出边 $(x, y, z)$,若 $dist[y] > dist[x] + z$,则更新 $dist[y] = dist[x] + z$。(对应于 RELAX 操作)
    \item 重复上述三个步骤,直到所有的节点都被标记。
\end{enumerate}

$Dijkstra$ 算法基于贪心思想,它只适用于所有边的权值都是非负数的图。

在第 2 步选出的 $x$,其 $dist[x]$ 已经是起点到 $x$ 的最短距离。我们不断选择全局最小值进行标记和扩展,最终可得到起点到每个节点的最短路长度。

\subsection{最小生成树问题}

最小生成树问题主要有两个方法:$Prim$ 算法和 $Kruskal$ 算法。这两个算法都基于一个推论:

给定一张无向图 $G=(V,E),n = |V|, m = |E|$。从 $E$ 中选出 $k < n-1$ 条边构成 $G$ 的一个生成森林。若再从剩余的 $m-k$ 条边中选 $n-1-k$ 条边添加到生成森林中,使其称为 $G$ 的最小生成树,\textbf{则最小生成树一定包含这 $m-k$ 条边中权值最小的边,且该边的两个端点不在同一颗树中}。

\subsubsection{Prim 算法}

任意时刻,设已经确定属于最小生成树的节点集合 $T$(起初只有 1 号点),剩余节点集合为 $S$,$Prim$ 算法每次需要找到一条边 $i$,它的边权 $z_i$ 满足:
$$z_i = \min_{(x,y,z) \in E, x\in S, y\in T}z$$

即该边两个端点分别属于集合 $S$ 和 $T$,并且权值最小。然后 $Prim$ 算法将 $x$ 点从 $S$ 中删除,加入到 $T$ 集合,并把 $z$ 累加到答案中。

在过程中,我们维护一个数组 $d$:

\begin{itemize}[noitemsep]
    \item 若 $x\in S$,则 $d[x]$ 表示 $x$ 与集合 $T$ 中的节点之间最小的边权。
    \item 若 $x \in T$,则 $d[x]$ 表示 $x$ 被加入 $T$ 时选出的最小边的边权。
\end{itemize}

然后再用一个数组 $v$ 标记节点是否属于 $T$ (即是否被选过),每次从未标记的点中选一个 $d$ 值最小的,把它标记(新加入 $T$),同时扫描所有出边,更新 $S$ 中其他点的 $d$ 值。

最后生成树的权值之和就是 $\sum_{x=1}^n d[x]$。

\subsubsection{Kruskal 算法}

Kruskal 算法是一种用来查找最小生成树的算法,由Joseph Kruskal在1956年发表。用来解决同样问题的还有Prim算法和Boruvka算法等。三种算法都是贪心算法的应用。和Boruvka算法不同的地方是,Kruskal算法在图中存在相同权值的边时也有效。

该算法的步骤如下:

\begin{enumerate}[itemsep=2pt,topsep=0pt,parsep=0pt]
    \item 新建图 $G$, $G$中拥有原图中相同的节点,但没有边
    \item 将原图中所有的边按权值从小到大排序
    \item 从权值最小的边开始,如果这条边连接的两个节点于图$G$中不在同一个连通分量中,则添加这条边到图$G$中
    \item 重复3,直至图$G$中所有的节点都在同一个连通分量中
\end{enumerate}

\begin{figure}
    \centering
    \includegraphics[width=12cm]{img/lab6/kruskal.pdf}
    \caption{kruskal 求解过程}
    \label{fig:kruskal}
\end{figure}


图 \ref{fig:kruskal} 展示了 Kruskal 的求解过程。Kruskal 的时间复杂度为 $O(m\log m)$,其中对边进行排序的复杂度为 $O(m\log m)$。依次遍历每条边,判断两端点是否属于同一个连通分量的复杂度可以非常小,使用并查集实现。

\subsection{并查集}
\begin{figure}[h]
    \centering
    \includegraphics{img/lab6/并查集.pdf}
    \caption{并查集结构}
    \label{fig:uf}
\end{figure}

并查集是一种树形的数据结构,顾名思义,它用于处理一些不交集的\textbf{合并}及\textbf{查询}问题。 它支持两种操作:

\begin{enumerate}[itemsep=2pt,topsep=0pt,parsep=0pt]
    \item 查找(Find):确定某个元素处于哪个子集;
    \item 合并(Union):将两个子集合并成一个集合。
\end{enumerate}

在上图中,分别有两个集合 $\{1, 2, 3, 4, 5\}$ 和 $\{6, 7, 8\}$,如果用 $f[x]$ 表示$x$在树中的父亲,那么就有 $f[2] = 1, f[4] = 3, f[8] = 7$。特殊的,令树根的父亲是树根自己,即 $f[1] = 1, f[6] = 6$。

在判断某两个元素是否属于同一集合时,首先分别找到其所在集合的根节点,如果根节点是同一个,那么表示这两个元素属于同一集合,否则表示这两个元素属于不同的集合。例如,在判断 4 和 5 是否属于同一集合时,分别通过递归找到其根节点都为 1,所以它们属于同一集合。再例如,判断 3 和 7 是否属于同一集合时,分别找到它们的根节点是 1 和 6,1 不等于 6,所以它们属于不同的集合。

\subsubsection{初始化}

一般初始化时,我们让所有点都指向自己,表示每个点都是一个独立的集合。
\begin{lstlisting}
    vector<int> f(n);
    for(int i = 0; i < n; i++) {
        f[i] = i;
    }
\end{lstlisting}
\subsubsection{查找}

\begin{figure}[h]
    \centering
    \includegraphics{img/lab6/并查集_get.pdf}
    \caption{在并查集中查找 4 的根结点}
    \label{fig:uf_get}
\end{figure}

在并查集中查找 4 的根节点的过程:

\begin{enumerate}[itemsep=2pt,topsep=0pt,parsep=0pt]
    \item $f[4] = 3$,继续查找 3 的根节点
    \item $f[3] = 1$,继续查找 1 的根节点
    \item $f[1] = 1$,1 就是根节点,返回结果 1
\end{enumerate}

代码如下:
\begin{lstlisting}
    int find(int x) {
        return x == f[x] ? x : find(f[x]);
    }
\end{lstlisting}

\subsubsection{路径压缩}

由于并查集仅支持合并和查询,在整个过程中集合不会被拆解,树只会在原来的基础上增加。因此,为了加快查找速度,我们会在每次查询结束后对路径进行压缩。

\begin{figure}[h]
    \centering
    \includegraphics{img/lab6/并查集_路径压缩.drawio.pdf}
    \caption{并查集的路径压缩}
    \label{fig:uf_cmpression}
\end{figure}

比如,在查询 4 的根节点时,会得到结果 1,紧接着直接修改 $f[4] = 1$。这样做的好处是接下来再次查询时会快很多,省去了多余的查询。从下面的代码中,可以看出我们并不只是对 4 进行修改,而是对从 4 到 1 的路径上的所有节点都进行了修改,正所谓「路径压缩」。

\begin{lstlisting}
    int find(int x) {
        return x == f[x] ? x : f[x] = find(f[x]);
    }
\end{lstlisting}

把在路径上的每个节点都直接连接到根上,这就是路径压缩。

\subsubsection{合并}

\begin{figure}[h]
    \centering
    \includegraphics{img/lab6/并查集_union.pdf}
    \caption{并查集的合并}
    \label{fig:uf_union}
\end{figure}

在合并 $x$ 所在集合与 $y$ 所在集合时,我们首先利用 get 函数找到他们对应的根结点 $\textit{root}_x$ 和 $\textit{root}_y$,然后令 $f[\textit{root}_x] = \textit{root}_y$ 即可。

\subsubsection{时间复杂度}

你可以先不严谨的认为,并查集的单个操作都是 $O(1)$ 的。如果想要详细了解相关内容,可以参考:\href{https://zh.wikipedia.org/wiki/并查集}{https://zh.wikipedia.org/wiki/并查集}

\section*{更新历史}

\begin{center}
    \begin{tabular}{|c|c|c|}
        \hline
        \textbf{更新时间} & \textbf{助教} & \textbf{更新内容} \\
        \hline
        2024 & 黄韦杰, 刘嘉杰 & 修改与添加拓展题描述\\
        2023 & 王梓健, 黄韦杰 & 添加 OJ 相关表述,完善本地环境使用方法 \\
        2022 & 邢广杰, 李晓晓 & 完善实验题目,开始更新记录 \\
        2021 & 韩耀东 & 主要内容构建 \\
        % Add more rows as needed
        \hline
        \end{tabular}
\end{center}

\end{document}