\documentclass[12pt,a4paper]{article}
%-- coding: UTF-8 --
\usepackage[UTF8]{ctex}
\usepackage[utf8]{inputenc}
\usepackage{geometry}
\usepackage{graphicx} % 引入图片
\usepackage{enumitem} % 取消列表默认间距
\geometry{left=3.18cm,right=3.18cm,top=2.54cm,bottom=2.54cm}
\usepackage{hyperref}
\hypersetup{hidelinks,
	colorlinks=true,
	allcolors=black,
	pdfstartview=Fit,
	breaklinks=true}
\usepackage{listings}
\usepackage{xcolor}
\usepackage{fontspec}
\usepackage{booktabs} % 三线表

\usepackage{tikz}
\usepackage{amsmath}
\usepackage{colortbl}
\newcommand\y{\cellcolor{clight2}}
\definecolor{clight2}{RGB}{212, 237, 244}%
\newcommand\tikznode[3][]%
   {\tikz[remember picture,baseline=(#2.base)]
      \node[minimum size=0pt,inner sep=0pt,#1](#2){#3};%
   }
\tikzset{>=stealth}
\renewcommand\vec[1]{\mathbf{#1}}

% 嵌入代码风格
\lstset{
	language    = c++,
	breaklines  = true,
	captionpos  = b,
	tabsize     = 4,
	columns     = fullflexible,
	commentstyle = \color[RGB]{0,128,0},
	keywordstyle = \color[RGB]{0,0,255},
	basicstyle   = \small\ttfamily,
	stringstyle  = \color[RGB]{148,0,209}\ttfamily,
	rulesepcolor = \color{red!20!green!20!blue!20},
	showstringspaces = false,
}



\title{vector 的基本使用}

\author{}
\date{November, 2023}

\begin{document}
\maketitle

vector 可以被简单的看做是一个动态数组,可以像普通的数组一样使用 $[]$ 运算符来访问其中的元素。与普通数组不同的是,它可以方便的进行创建、调整大小、添加或删除元素等等。
\begin{lstlisting}
    #include <vector> // 引入头文件
    // 1. 定义方法:
    vector<int> a; // 定义一个元素类型为 int 的动态数组 a
    vector<int> b(10); // 定义一个包含 10 个 int(默认为 0) 的动态数组 b
    vector<int> c(10, 100); // 定义一个包含 10 个 100 的动态数组 c
    // 2. 访问元素
    int x = c[0]; // 访问 c 中下标为 0 的元素
    // 3. 获取数组大小
    int n = c.size();
    // 4. 判断数组是否为空
    if(a.empty()) {
        // 为空则 if 条件成立
    }
    // 5. 插入元素
    a.push_back(x); // 把 x 插入到 a 的尾部
    // 6. 删除元素
    a.pop_back(); // 删除 a 的最后一个元素
    // 7. 清空数组
    c.clear();
\end{lstlisting}

有关更多内容,你可以参考:

\begin{itemize}[noitemsep]
    \item \href{https://www.runoob.com/w3cnote/cpp-vector-container-analysis.html}{https://www.runoob.com/w3cnote/cpp-vector-container-analysis.html}
    \item \href{https://en.cppreference.com/w/cpp/container/vector}{https://en.cppreference.com/w/cpp/container/vector}
    \item \href{https://www.w3cschool.cn/cpp/cpp-i6da2pq0.html}{https://www.w3cschool.cn/cpp/cpp-i6da2pq0.html}
    \item \href{http://c.biancheng.net/view/6749.html}{http://c.biancheng.net/view/6749.html}
\end{itemize}


\end{document}